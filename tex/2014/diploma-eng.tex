\documentclass[a4paper,12pt]{article}
\usepackage{url}
\usepackage{listings}
\author{Babken Vardanyan}
\title{Software protection against reverse engineering}
\date{}

\begin{document}
%\maketitle

I would like to express deepest gratitude to my guide Robert Hakobyan
(SEUA) who has encouraged and supported me during the project work.

\hfill \hfill Babken Vardanyan

\newpage

\tableofcontents
\newpage

\section{Introduction}
Use of software applications has become one of the cornerstones of our lives.
With the spreading use of networks, there has been remarkable progress in the
flow configuration of programs and digital content.

Applications can be quite expensive and thus illegal use of software emerged.
Accompanying this, there is an increasing demand for techniques which can
prevent internal analysis and tampering with programs by end users.

Most of the software today is commercial and needs protection against threats
such as reverse engineering, tampering and piracy. Protection against
malicious users is a hard problem, since the user is in absolute control of
his machine. Nevertheless, software owners manage to arm themselves against
these threats.

From the perspective of a software company, it is highly desirable that the
company's products are difficult to pirate and reverse engineer. Making
software difficult to reverse engineer seems to be in conflict with the idea
of being able to recover the software's design later on for maintenance and
evolution. Therefore, software manufacturers usually don't apply anti-reverse
engineering transformations to software binaries until it is packaged for
shipment to customers. Software manufacturers will typically only invest time
in making software difficult to reverse engineer if there are particularly
interesting algorithms that make the product stand out from the competition.
Making software difficult to pirate or reverse engineer is often a moving
target and requires special skills and understanding on the part of the
developer. Software developers who are given the opportunity to practice
anti-reversing techniques might be in a better position to help their
employer, or themselves, protect their intellectual property. As they say,
"to defeat a crook you have to think like one." By reverse engineering
viruses or other malicious software, programmers can learn their inner
workings and witness first-hand how vulnerabilities find their way into
computer programs.

This thesis examines code obfuscation techniques to protect software against
analysis and unwanted modifications. Program obfuscation makes code harder
to analyze. Indirectly, this also contributes to protecting against malicious
modifications to a program. This stems from the fact that an attacker first
must understand the software before he can make specified modifications. In
addition to techniques that improve a program’s analysis resistance, one can
add techniques that make tampering hard.

\subsection{Machine code}
Machine code or machine language is a system of instructions and data directly
understandable by a computer's central processing unit.
The executable representation of software, otherwise known as machine code,
is typically the result of translating a program written in a high-level
language, using a compiler, to an object file, a file which contains
platform-specific machine instructions. The object file is made executable
using linker, a tool which resolves the external dependencies that the object
file has, such as operating system libraries. In contrast to high-level
languages, there are low-level languages which are still considered to be
high-level by a computer's CPU because the language syntax is still a textual
or mnemonic abstraction of the processor's instruction set. For example,
assembly language, a language that uses helpful mnemonics to represent
machine instructions, still must be translated to an object file and made
executable by a linker.

However the translation from assembly code to machine
code is done by an assembler instead of a compiler-reflecting the closeness
of the assembly language's syntax to actual machine code. The reason why
compilers translate programs coded in high-level and low-level languages to
machine code is three-fold: CPUs only understand machine instructions, having
a CPU dynamically translate higher-level language statements to machine
instructions would consume significant, additional CPU time, and a CPU
that could dynamically translate multiple high-level languages to machine
code would be extremely complex, expensive, and cumbersome to
maintain-imagine having to update the firmware in your microprocessor every
time a bug is fixed or a feature is added to the C++ language! 

\section{Summary}
Advances in technology have led to the use of simple to use automated
debugging tools which can be extremely helpful in troubleshooting problems in
code. However, a malicious attacker can use these same tools. Securely
designing software and keeping it secure has become extremely difficult. 
These same easy to use debuggers can be used to bypass security built into
software. While the detection of an altered executable file is possible, it
is not as easy to prevent alteration in the first place. One way to make
alteration more difficult is to make analysis more difficult for the attacker.

First, we present a comprehensive overview of software protection techniques.
These are compared in terms of protection against analysis and protection
against modifications. In both cases a distinction is made between protection
against static attacks and protect against dynamic attacks.

This thesis presents an obfuscation method in which the program analysis by a
malicious user (attacker) is made difficult by shuffling section names
contained in PE header. In the proposed method, the section names are shuffled
in order to confuse the attacker.

Even if the attacker completely understands this method, it would still require
some time for him to understand the role of each section, since the order in
which section names are assigned is chosen randomly every time.

In order to make the analysis a success, the attacker must look at the
characteristics of each section and analyse their contents in order to
understand their purpose, which requires additional time from the attacker.

The proposed method can easily be automated, and requires no additional input
or configuration from software authors. This method does not have any of the
negative effects listed in section 4.4.

\section{Reverse engineering}

There are three major threats to software: reverse engineering, tampering, and
piracy.

\begin{itemize}
\item Software Reverse Engineering is the practice of analyzing a software 
    system, either in whole or in part, to extract design and implementation
    information in case of absense of source code. It represents techniques to
    inspect the inner workings of software applications.
\item Tampering covers ways to tamper with software. It is the practice of
    modifying parts of software in order to fit attacker's needs.
\item Piracy concerns unauthorized use of software.
\end{itemize}

The reasons for reverse engineering are:
\begin{enumerate}
\item Making illegal copies of software by bypassing validity checks
\item Studying viruses and other malware for security purposes
\item Competitive intelligence against companies working in the same field
    by extracting algorithms
\item Maintanance and creation of documentation for legacy software
\item Verification that implementation matches design 
\end{enumerate}

Reverse engineering consists of 2 steps: Analysis and Tampering.

\subsection{Analysis of the target}
During analysis, the attacker makes an attempt to discover algorithms, secret keys and other important information which can help him during the attack.

Typically, such an action is assumed to involve the following steps. First the
attacker disassembles the program and tries to understand the resulting
assembly program. However, a tremendous amount of labor and time is required
to understand the entirety of a large-scale program, which is not realistic.
Consequently, the attacker restricts the range to be considered (the range
which seems to be related to the secret information), and tries to understand
only that range. The restriction and understanding of the range is repeated
until the desired secret information is acquired.

The analysis can be static or dynamic.

\subsubsection{Static analysis}
Static analysis is done against non-executing code. In this case the attacker
use tools called Disassemblers, which translate the machine code present in the
file to human readable assembly language. This method is called "linear sweep".
An example of this method is a tool called objdump from GNU binutils.

The problem with this method is that it can be easily confused by using simple
anti-debugging methods like "junk byte insertion", which make the disassembler
go "off track" as it doesn't take into account the control flow of the program.

\subsubsection{Dynamic analysis}
During dynamic analysis they use tools called Debuggers. Most popular debuggers
are IDA Pro and OllyDBG. A technique called "recursive traversal" is used
during dynamic analysis, which takes control flow into account. However, as
some branches are input-dependent, usually not all target addresses can be
statically derived and disassembled. Debuggers analyse the control flow of the
code of the target and also allow the attacker to dynamically view and modify
the memory and CPU registers of the program while executing. 

This method is much slower than Static analysis as the debugger must "sweep"
every possible execution path. This method can also be confused by many    
anti-debugging techniques, and doesn't guarantee correct results.

\subsection{Tampering}
Typically,
one first needs information on the program internals before one can successfully
tamper with the software. Therefore, tampering attacks are most often preceded
by several reverse-engineering techniques. And once sufficient analysis has been
performed, the attacker will try to change the software to meet his own needs.
A hacker can make software available for everyone, even without authentication
or rights to use the software. Similar to analysis, tampering can also be done
statically or dynamically, this means on static or on executing code.

\subsubsection{Static tampering}
Static tampering techniques modify a static binary image.
An automated attack consisting of downloading a crack and applying it on a
stored binary is usually a static tampering attack, assuming that code is not
loaded in memory and modified there.

\subsubsection{Dynamic tampering}
Dynamic tampering techniques modify an application
at run-time. First, debuggers load code in memory. When stepping through
instructions, one can modify code, data, or the state (e.g. register values) of
the loaded program. A dynamic tampering attack is often performed “by hand”
and has similarities to software debugging.

\section{Protection techniques}
The need for software protection originates in multiple fields. Data security
covers the confidentiality and integrity of data, typically during transmission
and storage. Software security is the study of protecting software against
analysis, tampering, and other means of exploitation. We will focus only on
software security.

We follow the view where applications consist of functional code (performing
the actual program task), extended with other code (e.g. security code,
error-handling code, etc.). Often one or more components are critical to the
"normal" execution of the program, i.e. correct and legitimate execution of
the code. Software protection techniques serve as a binding to glue these
critical components with the rest of the code into one monolithic application,
whereas without software protection these types of code would have been easy to
identify, isolate, and attack.

There are different protection methods, but none of them guarantees 100\%
protection, but infact these only increase the attack time.

Software can be protected in several principles:

\begin{enumerate}
\item Watermarking
\item Software as service
\item Obfuscation
\end{enumerate}

We will focus only on Obfuscation.

\subsection{Obfuscation}
Obfuscation is a set of program transformations
that make program code and/or program execution difficult to analyze.
Obfuscation is the most popular protection method. It is designed to confuse
the attacker, make the work of debuggers more difficult and makes assembly
code less understandable by covering the program's actions while keeping the
algorithms intact.

The definition of Obfuscation is: let's pretend that P software code becomes
P' after some changes. That $P -> P'$ modification is called
obfuscating modification if P and P' behave identically.

\subsection{Protection against static attacks}
\subsubsection{Polymorphism}
    Polymorphic code is code that mutates while keeping the original algorithm
    intact. This technique is sometimes used by computer viruses, shellcodes
    and computer worms to hide their presnece. Most anti-virus software and
    intrusion detection systems attempt to locate malicious code by searching
    through the computer files and data packets sent over a computer neetwork.
    If the security software finds patterns that correspond to known computer
    viruses or worms, it takes appropriate steps to neutralize the threat.
    Polymorphic algorithms make it difficult for such software to locate the
    offending code as it constantly mutates.

\subsubsection{Self-modifying code}
    Self-modifying code is code that modifies itself on purpose.
    Self-modifying code was used in the early days of computers in order to
    save memory space, which was limited. It was  used to implement subroutine
    calls and returns when the instruction set only provided simple branching
    or skipping instructions to vary the flow of control.

    Nowadays, self-modifyingn code is used by programs that do not want to
    reveal their presence, such as computer viruses and some shellcodes.
    Shellcode is a relocatable piece of machine code used as the payload in
    the exploitation of a software bug which allows an unauthorised user to
    communicate with the computer via the operating system's command line
    as a result of exploiting a vulnerability in software running on the
    machine.

    Viruses and shellcodes that use self-modifying code, also mostly do this
    in combination with polymorphic code. Polymorphic viruses are sometimes
    called primitive self-mutators. Modifying a piece of running code is
    also used in certain attacks, such as buffer overflows.

\subsubsection{Encryption}
    This is one of the simplest ways is encryption of data. When data is openly
    stored in data section, it can give some clues to the attacker about the
    actions performed in the program.

    Encryption is the process of obscuring information to make it unaccessable
    without special knowledge. Encryption can be used to ensure secrecy, but
    other techniques are still needed to verify the integrity and authencity.
    Encryption or software code obfuscation is used in software copy protection
    against reverse engineering, unauthorized application analysis and software
    piracy. It is used in different encryption or obfuscating software.

    Encryption is most commonly used method of achieving polymorphism in code.
    However, not all of the code can be encrypted as it would be completely
    unusable. A small portion of it is left unencrypted and used to jumpstart
    the encrypted software. Anti-virus software targets this small
    unencrypted portion of code. Malicious programmers have sought to protect
    their polymorphic code from this strategy by rewriting the unencrypted
    decryption engine each time the virus or worm is propagated. Sophisticated
    pattern analysis is used by anti-virus software to find underlying
    patterns within the different mutations of the decryption engine in hopes
    of reliably detecting such malware.

\subsubsection{Metamorphism}
    Metamorphic code is code that can reprogram itself. Often it does this by
    translating its own code into a temporary representation, and then back
    to normal code again. This is used by some viruses when they are about to
    infect new files, and the result is that their children will never look
    like themselves. The computer viruses that use this techineque do this in
    order to avoid the pattern recognition of anti-virus software - the actual
    algorithm does not change, but everything else might.

    Metamorphic code is more effecient than polymorphic code. This is because
    most anti-virus software willl try to search for known virus-code even
    during the execution of the code.

    Metamorphic code can also mean that a virus is capable of infecting
    execuables from 2 or more different operating systems (such as Windows and
    Linux) or even different computer architectures. Often, the virus does
    this by carrying several viruses itself, so it is really a matter of
    several viruses that have been joined together into a "supervirus".

\subsubsection{Packers}
    These programs are more often used to compress
    executable files, but can also increase difficulty for the less skilled
    attackers. Most popular packer currently is UPX.

    Executable packing is the process of compressing an executable file and
    prepending a decompression stub, which is responsible for decompressing
    the executable and initiating it's execution. The decompression stub is a
    standalone executable, making packed and unpacked executables
    indistinguishable to the casual user as they are not required to perform
    any additional steps to start execution.
    
    Executable compression is any means of compressing an executable file and
    combining the compressed data with the decompression code it needs into a
    single executable. When executed, this will unpack the original executable
    code and transfer control to it. The effect is the same as if the original
    uncompressed executable had been run. A compressed executable requires:
    \begin{itemize}
    \item Less storage space in the file system
    \item Less time to transfer data from the file system into memory
    \item More time to decompress the data before execution begins than the
        uncompressed original
    \end{itemize}

\subsubsection{Junk byte insertion}
    One of the simplest methods, it makes static disassembly more
    difficult as disassemblers have no way to differentiate between junk
    byte and a CPU instruction. This is easily done with inline assembly
    blocks. This method does nothing against recursive debuggers, which
    do look at execution path of the program.

\subsection{Protection against dynamic attacks}

\subsubsection{Protectors}
An executable protector is in fact a derive of the simple packer, meaning
that protector emphasises more or even completely on protection against
reverse engineering than the simple packer.

This however brigns some important disadvantages to the distributed
program: first of all the size. Where packers aim to reduce the packed
executable's size as much as possible, the protector often adds so much
code trying to protect against reverse engineering, that the size of the
total can considerably increase. There are examples where the size
increased to about 600\% where the simple packer reduced the size to about
30\% of the original.

\subsubsection{Code execution path obfuscation}
This method includes:
\begin{itemize}
\item high level obfuscation
\item randomization of instruction order
\item inserting new code blocks, which do not affect execution of the
    program
\item using inline functions instead of actual function calls
\item inserting duplicate functions, which perform same operations in
    different algorithms
\item duplicating loop body
\item Branching functions\\
    Instead of jumps, a function is called, which performs a jump but never
    returns back, unlike a regular jump.
\end{itemize}

\subsubsection{Checking for Debuggers present in the system}
If the program detects a debugger process running in the system or on the
hard drive, it can perform appropriate actions to confuse the attacker.
Presense of debuggers can be checked by monitoring CPU's debug registers
or detecting changes in code segment (int 3 instruction insertino).

Another example of this method is the IsDebuggerPresent() function of Win32
API. It determines whether the calling process is being debugged by a
user-mode debugger.

\subsubsection{Hiding parts of the program}
Some software authors decide to hide some of their code, for example in
external devices (hardware assisted solutions) or web servers.

\subsubsection{Dynamic self-modifying code}
During this method modifications are done in code segment while executing
that code. This is often used in viruses.

\subsubsection{Parallelization}
In linear programs it is easy to guess the execution path. When using this
method, program is divided into several parts which do not depend on each
other. Then these parts are executed separately in different threads or
processes. This exponentially increases possible execution paths and time
needed to analyse the code.

\subsubsection{Virtualization}
Virtualizing program execution in a virtual machine

\subsection{Negative effects}
When selecting a protection method, the author of the software must keep
in mind that most of these methods have negative effects on your software:
\begin{enumerate}
\item Demand additional resources (time, financial) for software maintenance
\item Make debugging more difficult
\item Make program slower
\item Increase executable file size
\item Make the program harder to read and understand
\item The less of your project cost is being invested in adding value for the
    customer
\item The more expensive your product becomes, driving away legitimate
    customers
\item The longer your time to market, allowing a less conservative competitor
    to execute faster
\end{enumerate}

These methods are ususally used in combination, but in that case their
negative effects are added together. That's why it's important to keep in
mind these effects when using protection.

\subsection{Evaluating protection}
We need ways to evaluate effectivness of protection methods described
above. The following numbers are used for that:
\begin{enumerate}
\item Potency\\
    Shows how complicated it is for a skilled attacker to understand the code.
    It is assumed that the more complicated the program is, the higher is the
    protection.
\item Flexibility\\
    Shows how well the program resists automated attacks.
\item The price of protection\\
    Shows how much the execution time and file size increase when using a
    protection method.
\item Stealth\\
    Shows how well some data or a part of the code can be protected from a
    skilled attacker.
\end{enumerate}

\section{PE file structure}
Portable Executable (PE) is a file format for executables, object code and
DLLs. It was designed by Microsoft in early 90s. It consists of a header
followed by data structures (sections). PE files are not loaded into memory as
they appear on disk, instead being loaded according to the information in
the PE file header. Unlike MS-DOS, segment registers are not used to relocate
code and data, with ‘sections’ instead being used to hold program code,
variable, constant and resource data. Where possible, the PE file format
stores these sections in a packed state, holding information as to how much
memory will be required as opposed to the actual data itself. Also unlike
MS-DOS, Windows assigns a 4GB address space to each newly loaded program.

In reverse engineering and software protection, it is important to understand
the PE file structure. It is even more imoprtant when dealing with
packers/unpackers and when adding code to executable (keygens, other
functionality etc).

\subsection{Header}
The header is 0x1000 bytes long and consists of:

\subsubsection{DOS header and stub}
DOS header and DOS stub are ignored by Windows loader, except:

\begin{itemize}
\item the first 2 bytes are the EXE signature. They must be "MZ" in order
    the file to be registered as valid .exe
\item the last part of DOS header, the dword at 0x3C. It is the offset to
    PE header, relative to the file beginning. It is a pointer to PE file
    header, just past the DOS stub
\end{itemize}

The code in stub is executed only if the executable is ran under DOS or from
command line. In that case, it usually prints a message like "This program
cannot be run in DOS mode." and exits.

DOS header is 0x40 bytes long, and the size of DOS stub depends on the offset
to PE header.

\subsubsection{PE header}
PE header begins with 4-byte signatue - it must be
"\texttt{PE\textbackslash0\textbackslash0}" in order the file to be registered
as a valid .exe.

The signature is followed by File Header which contains information about the
physical structure and properties of the file. It is 20 bytes long and
consists of:

\begin{itemize}
\item Target machine architecture
\item Number of sections
\item Date \& time stamp
\item Pointer to symbol table (0 for image)
\item Number of symbols (0 for image)
\item Size of optional headers
\item Characteristics
\end{itemize}

\subsubsection{Optional header}
Next comes the Optional Header which contains info about the logical layout
inside the PE file. This section is 224 bytes long, the last 128 of which
is the Data Directory. This section is always required to be present in every
.exe file, despite the name.

Optional header contains information such as:

\begin{itemize}
\item Versions of linker, OS, image, subsystem
\item Image base address
\item Base address and size of code and data
\item Entry point address
\item Size of stack, image and headers
\item Section alignment in file and memory
\item File checksum
\item Flags and characteristics etc
\end{itemize}

\subsubsection{Data directories}
Technically a part of Optional header, this structure is an array of 16
IMAGE\_DATA\_DIRECTORY structures. Each structure is 8 bytes each and
contains info about a predefined item, such as import table. Each structure
consits of 2 dwords, which contain the relative virtual address (RVA) and
size of the data structure in question.

Directories contain data like import and export tables, resources, exception
handling info, debug, copyright sections etc.

\subsubsection{Section table}
Section table is an array of IMAGE\_SECTION\_HEADER structures, each containing
information about one section in the PE file, such as:
\begin{itemize}
\item Section name, 8 bytes long
\item Location and size of the section on disk
\item Virtual address at which the section should be loaded
\item Characteristics
\item Relocation and line number info
\end{itemize}

Section names are irrelevant, but for programmers'
convenience they are usually called

\begin{itemize}
\item .text - Executable code
\item .data - Readable and writeable data
\item .bss - Uninitialized data
\item .rdata - Read only data
\item .edata, .idata - Export and Import data sectioins
\item .rsrc - Windows resources (icons, images, text strings etc)
\item .src, .debug - Debug information
\item .reloc - Base relocations; contains info about where the linker
    assumed the sections will be loaded, so the addresses can be fixed.
\end{itemize}

\subsection{Sections}
The header is followed by sections which contain actual code, data, resources
etc.

In the file on disk, each section starts at an offset that is multiple of
FileAlignment value of Optional Header. Between each section there's 00 byte
padding. When loaded into memory, the sections always start on a page boundry
so that the first byte of each section corresponds to a memory page. On x86
CPUs SectionAlignment is 4kB.

These data structures of PE files on disk are the same data structures when
they are loaded into memory. When executed, Win32 loader simply maps those
data structures into memory.

Each thread sees memory as a 4GB array of bytes. In practice only some portions
of this array are visible and accessible by our program.

\subsection{Windows Loader}
When a PE-format file is to be executed, the file handle is passed to the
CreateProcess() API function and statically imported DLL files are loaded
(EXE and DLL files are both in PE format and are identical apart from a
single-bit flag). This initialization is performed by the OS component called
the loader, detailed in this section. The loader reads the PE file header and
copies it to the 32-bit ImageBase address taken from the header; if this 
address is in use the loader determines an alternative address. It then reads
each section and copies each one to memory. Two distinct alignments may be
specified: file alignment and section alignment. If these alignments are
different, the loader copies each section to memory according to the section
alignment, which in most cases results in the memory layout being different
from that of the PE file. If the loader uses a different ImageBase address
from that specified in the header - as is often the case with .dll files -
address relocation must be performed. Relative jumps and calls do not need to
be changed, but all absolute addresses (calculated during the linking stage
of program compilation) must be adjusted by taking the difference between
the preferred ImageBase and the chosen ImageBase and altering the addresses
accordingly. Nearly all programs import DLLs of some kind. Regular Windows
executables are unable to call int instructions (as was possible with
MS-DOS) and must make API calls in order to perform useful work. The loader
loads the statically imported DLLs, resolves the APIs and writes the
addresses of API calls to memory. This process is described in more detail
in the following section. With these processes complete, the process memory
layout differs from that of the PE file in terms of alignment, section
length, address relocation and API resolution. If program execution has begun
the variable data section will also be tainted.

\section{Protection by shuffling section names}
This thesis presents an obfuscation method in which the program analysis
by a malicious user (attacker) is made difficult by shuffling section names
contained in PE header.

The idea is to make it harder for the attacker to find the section in which
the modifications should be made. This is done by changing the section names
by shuffling them. Shuffling is done using Knuth's shuffle algorithm (see
next section).

First the program parses the PE header, then reads section names into an
array. Then after shuffling the array, section names are written back
into the file.

The program also prints the old and new section order to stdout.

Even if the attacker completely understands this method, it would still
require some time for him to understand the role of each section, since the
order in which section names are assigned is chosen randomly every time.
In order to make the analysis a success, the attacker must look at the
characteristics of each section and analyse their contents in order to
understand their purpose, which requires additional time from the attacker.

The proposed method can easily be automated, and requires no additional
input or configuration from software authors. This method does not have any
of the negative effects listed in section 4.4.

The implementation uses C language and data structures from \texttt{winnt.h}
header. The program works on both Windows and Linux systems.

It is important to keep in mind that this method, just like almost any other
protection methods, does not protect the executable file from modifications.
It just makes the analysis and tampering somewhat harder.

The following is the main function of the implementation:

\begin{verbatim}
int ParseFile(FILE *file)
{
    IMAGE_DOS_HEADER dosHeader;
    IMAGE_NT_HEADERS header;

    if (ReadDosHeader(file, &dosHeader)) {
        return 1;
    }

    if (ReadPeHeader(file, &header, dosHeader.e_lfanew)) {
        return 1;
    }

    int sectionCount = header.FileHeader.NumberOfSections;
    IMAGE_SECTION_HEADER sections[ sectionCount ];

    if (ReadSections(file, sections, sectionCount)) {
        return 1;
    }

    unsigned char *names[ sectionCount ];
    ReadSectionNames(sections, names, sectionCount);

    printf("Old order: ");
    PrintSectionNames(names, sectionCount);

    if (Shuffle(names, sectionCount)) {
        return 1;
    }

    printf("New order: ");
    PrintSectionNames(names, sectionCount);

    if (WriteSectionNames(file,
                          dosHeader.e_lfanew +
                          sizeof(IMAGE_NT_HEADERS),
                          names,
                          sectionCount)) {
        return 1;
    }

    return 0;
}
\end{verbatim}

\section{Knuth's shuffle algorithm}
The protection method described in this text uses Knuth's algorithm for
randomizing the order of section names.

Shuffling is a process where the order of elements in a data set is randomized
to provide an element of chance. One of the most common applications of this
is to shuffle a deck of cards.

Mathematically, the shuffling problem, is basically a problem of computing a
random permutation of N objects, since shuffling a deck of cards is basically
subjecting a deck of cards to a random permutation.

Knuth shuffle (after Donald Knuth), also known as Fisher-Yates shuffle, is an
algorithm for generating a random permutation of a finite set.

There are few basic well known solutions to this problem, the first of this is
an O(n lg n) algorithm. The solution involves simple assigning a random number
to each card, and sorting them in order of their assigned number. There is a
chance that two of the cards are assigned the same number. This check can be
checked each time you assign a number, or even better this can be checked when
you sort the cards and redo it all over again, or redo the same problem on the
smaller set usually 2 cards, if you have got more than 2 cards, actually even
2 cards, then you’ve chosen a bad random number generator.

The more elegant and faster of the two algorithms is also known as the Knuth
Shuffle, popularized by Donald Knuth, in his book , The Art of Computer
Programming. The algorithm was originally published by R.A. Fisher and F.
Yates [Statitiscal Tables (London 1938, Example 12], in ordinary language, and
by R. Durstenfeld [CACM 7 (1964), 420] in computer language. Here’s the
algorithm in ordinary language.

Let $X_{1}, X_{2}, ..., X_{N}$ (In this case $N=52$) be the set of N numbers
to be shuffled.

\begin{enumerate}
\item Set j to N
\item Generate a random number R. (uniformly distributed between 0 and 1)
\item Set k to $jR+1$. k is now a random integer, between 1 and j.
\item Exchange $X_{k}$ and $X_{j}$
\item Decrease j by 1.
\item If j \textgreater 1, return to step 2.
\end{enumerate}

The Fisher-Yates shuffle is an in-place shuffle. That is, given a
preinitialized array, it shuffles the elements of the array in place, rather
than producing a shuffled copy of the array. This can be an advantage if the
array to be shuffled is large.

The following is our implemetation of this algorithm for the project, from file
\texttt{shuffle.c}:

\begin{verbatim}
#include <string.h> // memcpy()
#include <stdlib.h> // malloc(), free(), rand()
#include <stdio.h>

#include "winnt.h"

static int Swap(unsigned char *a, unsigned char *b)
{
    void *temp = malloc(IMAGE_SIZEOF_SHORT_NAME);
    if (temp == NULL) {
        perror("Could not allocate memory");
        return 1;
    }

    memcpy(temp, a, IMAGE_SIZEOF_SHORT_NAME);
    memcpy(a, b, IMAGE_SIZEOF_SHORT_NAME);
    memcpy(b, temp, IMAGE_SIZEOF_SHORT_NAME);
    free(temp);

    return 0;
}

int Shuffle(unsigned char * arr[], int length)
{
    int random;

    int i;
    for (i = 0; i != length; i++) {
        random = rand() % (length - i);

        if (Swap(arr[ length - 1 - i ], arr[ random ])) {
            return 1;
        }
    }

    return 0;
}
\end{verbatim}

\section{Conclusions}
Since this protection method is applied on compiled executables, that means no
modification is performed on source code of programs. This in turn means that
none of the negative effects of software protection listed in section 4.4 are
present in our method.

This protection method can be used in conjunction with any other method, as it
does not modify anything except the section names in the PE header.

Since section names are generated randomly, this method is
resistant to break once-run everywhere types of attacks.

\section{References}
\url{https://en.wikibooks.org/wiki/X86_Disassembly/Windows_Executable_Files#PE_Optional_Header}\\
\url{http://www.accessroot.com/arteam/site/download.php?view.112}\\
\url{http://aerokid240.blogspot.com/2011/03/windows-and-its-pe-file-structure.html}\\
\url{http://www.stackexchange.com/}\\
\url{http://www.coep.org.in/page_assets/341/121003016.pdf}\\
\url{http://tekpool.wordpress.com/2006/10/06/shuffling-shuffle-a-deck-of-cards-knuth-shuffle/}

\end{document}

