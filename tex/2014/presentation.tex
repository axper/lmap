\documentclass[11pt]{beamer}
\usefonttheme{serif}
\usepackage{polyglossia}
\usepackage{fontspec}
\usepackage{xltxtra}
\usepackage{url}
\usepackage{listings}
\usepackage{graphicx}
\usepackage{xcolor}
\usepackage{hyperref}
\usepackage{tikz}
\setdefaultlanguage{armenian}
\newfontfamily{\armenianfont}{DejaVu Sans}
\newfontfamily{\armenianfonttt}{DejaVu Sans Mono}
\newfontfamily{\armenianmathfont}{DejaVu Sans}
\newfontfamily{\armenianfont}{DejaVu Sans}
%\ExplSyntaxOn
%\DeclareSymbolFont{armenianletters}{\g_fontspec_encoding_tl}{\l_fontspec_family_tl}{m}{it}
%\int_step_inline:nnnn { "531 } { 1 } { "556 }
%{
%\Umathcode #1 = "0 \symarmenianletters #1
%}
%\int_step_inline:nnnn { "561 } { 1 } { "587 }
%{
%\Umathcode #1 = "0 \symarmenianletters #1
%}
%\ExplSyntaxOff
\renewcommand*\contentsname{Բովանդակություն}
\renewcommand{\figurename}{Նկար}
\urlstyle{tt}
\hypersetup{
    colorlinks=true,
    linktoc=all,
    linkcolor=blue,
}
\definecolor{bluekeywords}{rgb}{0,0,1}
\definecolor{greencomments}{rgb}{0,0.5,0}
\definecolor{redstrings}{rgb}{0.64,0.08,0.08}
\definecolor{xmlcomments}{rgb}{0.5,0.5,0.5}
\definecolor{types}{rgb}{0.17,0.57,0.68}
\lstset{language=C++,
  frame=b,
  showtabs=false,
  tabsize=4,
  breaklines=true,
  showstringspaces=false,
  breakatwhitespace=true,
  escapeinside={(*@}{@*)},
  commentstyle=\color{greencomments},
  keywordstyle=\color{bluekeywords},
  stringstyle=\color{redstrings},
  basicstyle=\ttfamily
}
\definecolor{lightgray}{gray}{0.9}

\title[Մեքենայական լեզվով ծրագրերի պաշտպանությունը վերծանումից]{Մեքենայական լեզվով ծրագրերի պաշտպանությունը վերծանումից}
\author{Խումբ՝ Հ055-2\\Ուսանող` Բաբկեն Վարդանյան\\Ղեկավար՝ տ․գ․թ, դոցենտ Ռ․ Գ․ Հակոբյան}
\institute[ՀՊՃՀ]{Հայաստանի Պետական Ճարտարագիտական Համալսարան\\Քոմփյութերային Համակարգերի և Ինֆորմատիկայի Ֆակուլտետ}
\date{Երևան 2014}

\usetheme{Singapore}
\usecolortheme{seahorse}
\setbeamertemplate{frametitle}[default][center]


\begin{document}
\begin{frame}
\titlepage
\end{frame}

%\begin{frame}%\frametitle{Outline}
%\tableofcontents
%\end{frame}

\section{Ներածություն}

\begin{frame}\frametitle{Ներածություն}
\begin{itemize}
\item Ծրագրերի մեծամասնությունը պաշտպանության կարիք ունի
\item Պաշտպանությունը դժվար գործ է, բայց անհրաժեշտ
\item Հնարավոր չէ ծրագիրը 100\%֊ով պաշտպանել
\item Բայց հնարավոր է վերծանման ժամանակը ավելացնել
\end{itemize}
\end{frame}

\section{Վերծանում և պաշտպանություն}

\begin{frame}\frametitle{Սպառնալիքներ}
Ծրագրային ապահովմանը սպառնալիքներ
\begin{enumerate}
\item Վերծանում
\item Փոփոխություններ
\item Ապօրինի օգտագործում
\end{enumerate}
\end{frame}

\begin{frame}\frametitle{Վերծանման միջոցներ}
\begin{columns}[t]
\column{.5\textwidth}
\begin{block}{Ստատիկ վերծանում}
\begin{figure}[p]
\centering
\includegraphics[width=1.0\textwidth]{static.png}
\end{figure}
\end{block}
\column{.5\textwidth}
\begin{block}{Դինամիկ վերծանում}
\begin{figure}[p]
\centering
\includegraphics[width=1.0\textwidth]{olly.png}
\end{figure}
\end{block}
\end{columns}
\end{frame}

\begin{frame}\frametitle{Պաշտպանության միջոցներ}
Պաշտպանության միջոցների տեսակներ
\begin{enumerate}
\item Ջրանշում
\item Ծրագիրը որպես ծառայություն
\item Օբֆուսկացիա
\end{enumerate}
\end{frame}

\begin{frame}\frametitle{Օբֆուսկացիա}
\footnotesize{
\begin{columns}[t]
\column{.5\textwidth}
\begin{block}{\small{Ստատիկ հարձակման դեմ}}
\begin{itemize}
\item Ինքնաձևափոխվող կոդ
\item Գաղտնագրում
\item Սեղմող ծրագրեր
\end{itemize}
\end{block}
\column{.5\textwidth}
\begin{block}{\small{Դինամիկ հարձակման դեմ}}
\begin{itemize}
\item Կարգաբերիչների առկայության ստուգում
\item Բազմաձևություն (պոլիմորֆիզմ)
\item Զուգահեռացում
\end{itemize}
\end{block}
\end{columns}
}
\end{frame}

\begin{frame}\frametitle{Պաշտպանության գնահատում}
\begin{itemize}
\item Կարողություն` մարդու դեմ
\item Ճկունություն՝ ավտոմատացված ծրագրի դեմ
\item Տվյալների թաքցնում
\item Գին (Բացասական ազդեցություններ)
	\begin{enumerate}
	\item Ծրագրի սպասարկման վրա ծախսեր
	\item Արագագործության կորուստ
	\item Կարգաբերման բարդացում
	\item Ֆայլի չափի մեծացում
	\end{enumerate}
\end{itemize}
\end{frame}

\section{PE ֆայլ}

\begin{frame}\frametitle{PE ֆայլի կառուցվածքը}
\begin{tikzpicture}[remember picture,overlay]
\node[at=(current page.center)] {
\includegraphics[width=\paperwidth]{structure.png}
};
\end{tikzpicture}
\end{frame}

\section{Մշակված ալգորիթմի աշխատանքը}

\begin{frame}\frametitle{Պաշտպանության ալգորիթմը}
\begin{enumerate}
\item Սեկցիաների անունների հասցեները պահվում են զանգվածի մեջ
\item Խառնվում են Կնուտի ալգորիթմով
\item Արդյունքը հետ է գրվում ֆայլի մեջ
\end{enumerate}
\begin{columns}[t]
\column{.5\textwidth}
\begin{block}{\small{Պաշտպանությունից առաջ}}
\begin{figure}[p]
\centering
\includegraphics[width=1.0\textwidth]{before.png}
\end{figure}
\end{block}
\column{.5\textwidth}
\begin{block}{\small{Պաշտպանությունից հետո}}
\begin{figure}[p]
\centering
\includegraphics[width=1.0\textwidth]{after.png}
\end{figure}
\end{block}
\end{columns}
\end{frame}

\begin{frame}\frametitle{Կնուտի ալգորիթմ}
\begin{center}
	\texttt{
\begin{tabular}{|r|c|r|l|}\hline
\multicolumn{1}{|c|}{\textnormal{Միջ.}} &
\multicolumn{1}{|c|}{\textnormal{Պատահ.}} &
\multicolumn{1}{|c|}{\textnormal{Մնացածը}} &
\multicolumn{1}{|c|}{\textnormal{Արդյունքը}}
\\ \hline \hline
 & & 1 2 3 4 5 6 7 8 & \\ \hline
1–8 & 6 & 1 2 3 4 5 \textbf{8} 7 & \textbf{6} \\ \hline
1–7 & 2 & 1 \textbf{7} 3 4 5 8 & \textbf{2} 6 \\ \hline
1–6 & 6 & 1 7 3 4 5 & \textbf{8} 2 6 \\ \hline
1–5 & 1 & \textbf{5} 7 3 4 & \textbf{1} 8 2 6 \\ \hline
1–4 & 3 & 5 7 \textbf{4} & \textbf{3} 1 8 2 6 \\ \hline
1–3 & 3 & 5 7 & \textbf{4} 3 1 8 2 6 \\ \hline
1–2 & 1 & \textbf{7} & \textbf{5} 4 3 1 8 2 6 \\ \hline
\end{tabular}
\textnormal{Արդյունք՝ } 7 5 4 3 1 8 2 6
	}
\end{center}
\end{frame}

\section{}

\begin{frame}\frametitle{Օգտագործված գրականություն}
\begin{enumerate}
\item \url{http://aerokid240.blogspot.com/2011/03/windows-and-its-pe-file-structure.html}
\item \url{http://en.wikipedia.org/wiki/Fisher–Yates\_shuffle}
\item \url{http://www.ollydbg.de/}
\item Project URL: \url{github.com/axper/shuffle-pe-section-names}
\end{enumerate}
\end{frame}

\end{document}
